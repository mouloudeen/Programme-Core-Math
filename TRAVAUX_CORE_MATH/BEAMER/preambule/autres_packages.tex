%\usepackage{natbib}         % Pour la bibliographie
\usepackage{url}            % Pour citer les adresses web
\usepackage[T1]{fontenc}    % Encodage des accents
\usepackage[utf8]{inputenc} % Lui aussi
\usepackage[english]{babel} 
\usepackage{numprint}       % Histoire que les chiffres soient bien

\usepackage{amsmath}        % La base pour les maths
\usepackage{mathrsfs}       % Quelques symboles supplémentaires
\usepackage{amssymb}        % encore des symboles.
\usepackage{amsfonts}       % Des fontes, eg pour \mathbb.


\usepackage{cancel}
%
% Choose how your presentation looks.
%
% For more themes, color themes and font themes, see:
% http://deic.uab.es/~iblanes/beamer_gallery/index_by_theme.html
%
\usetheme{Warsaw}


%\usepackage[svgnames]{xcolor} % De la couleur

%%% Si jamais vous voulez changer de police: décommentez les trois 
%\usepackage{tgpagella}
%\usepackage{tgadventor}
%\usepackage{inconsolata}

%%% Pour L'utilisation de Python
\usepackage{minted}
\usemintedstyle{friendly}

\usepackage{graphicx} % inclusion des graphiques
\usepackage{wrapfig}  % Dessins dans le texte.

\usepackage{tikz}     % Un package pour les dessins (utilisé pour l'environnement {code})

\usetikzlibrary{arrows}
\newcommand{\poubelle}[1]{}
\usepackage[framemethod=TikZ]{mdframed}
\usepackage{algorithm}
\usepackage{algorithmic}
\usepackage{xcolor}

\usepackage{pgf, tikz}
\usepackage[all]{xy}% pour les schemas


\newtheorem{prop}{Proposition}
\newtheorem{defin}{Definition}
\newtheorem{theo}{Theorem}
\newtheorem{coroll}{Corollary}
\newtheorem{lem}{Lemma}
\newtheorem{profl}{Lemma's proof}
\newtheorem{proft}{Theorem's proof}
\newtheorem{profc}{Corollary's proof}
\newtheorem{prope}{Property}
\newtheorem{algo}{Algorithme}


\setbeamercovered{transparent} 