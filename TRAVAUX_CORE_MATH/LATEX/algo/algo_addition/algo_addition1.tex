\section{Addition Operators}

\subsection{Add112}

See algorithm \ref{algo:Add112}
\begin{lem}[Add112]\label{lem:add112}
Let $a$ and $b$  floating point numbers, with $\lvert a \rvert \ge \lvert b \rvert $, $s$ and $t$ result of $Add112(a,b)$ for the $4$ modes of rounding, considering that there is no \textbf{overflow} so :
\begin{itemize}
 \item  $(1)$ $s+t$ is exactly equal to $a+b$.              
\item  $(2)$ $ \lvert t \rvert \le 2^{-53}\lvert s \rvert $ for $RNDN$ and $ \lvert t \rvert \le 2^{-52}\lvert s \rvert $ for direct rounding modes.
\end{itemize} 
\end{lem}

\begin{proof} \color{-yellow}
$(1)$:\\
According to the calculation technique of  \textbf{Fast2Sum algorithm} proof's (\cite{muller2010handbook}), this proof shown that for the closest rounding mode ($RNDN$).\\
We suppose that $\lvert a \rvert \ge \lvert b \rvert $ and  there is no \textbf{overflow}.\\
 We suppose $a > 0$ and $b > 0$ (respectively  $a < 0$ and $b < 0$).\\
 We take $b = b_h + b_{\ell}$ with $b_h$ a multiple of $ulp(a)$ ,$\lvert b_{\ell}\rvert
< ulp(a)$ and  $b_{\ell}$ is of the same sign that $b$ (and $b_h$).\\



Either we have $0 \le \lvert b_{\ell} \rvert  < 2^{-53}. \lvert a \rvert$ or  $ 2^{-53}. \lvert a \rvert \le\lvert b_{\ell} \rvert \le  2^{-52}. \lvert a \rvert$.\\
If $0 \le \lvert b_{\ell} \rvert  < 2^{-53}. \lvert a \rvert$ for the modes of Rounding $RNDZ$, $RNDN$ and $RNDD$ (respectively $RNDZ$, $RNDN$ and $RNDU$) then $b_{\ell}$ will be ignored
otherwise $RNDU$(respectively  $RNDD$) $b_{\ell}$ is not ignore, so  $\circ (b) = b_h + ulp(a)$;\\
If $ 2^{-53}. \lvert a \rvert \le\lvert b_{\ell} \rvert \le  2^{-52}. \lvert a \rvert$ for the modes of Rounding $RNDZ$ and $RNDD$ (respectively $RNDZ$ and 
$RNDU$) then $b_{\ell}$ will be also ignored for $RNDN$(closest round) and $RNDU$(respectively $RNDN$ and $RNDD$).\\
The first case when $b_{\ell}$ is ignored so we have $\circ (b) = b_h $.
$\Rightarrow$:
\begin{itemize}
\item $s = \circ (a+b)$ $\Rightarrow$ $s = a + b_h$
\item $z = \circ (s-a)$ $\Rightarrow$ $z = a + b_h - a$  $\Rightarrow$ $z = b_h$
\item $t = \circ (b-z)$ $\Rightarrow$ $t = b - b_h$ we have that $b = b_h+b_{\ell}$ $\Rightarrow$ $t = b_{\ell}$
\end{itemize}
So we have that $s+t = a+b_h+b_{\ell}$ $\Rightarrow$ $s+t$ is exactly equal to $a+b$.\\


The second case when $b_{\ell}$ is not ignored so we have $\circ (b) = b_h +ulp(a)$.
$\Rightarrow$ :
\begin{itemize}
    \item $s = \circ (a+b)$ $\Rightarrow$ $s = a + b_h + ulp(a)$
    \item $z = \circ (s-a)$ $\Rightarrow$ $z = a + b_h + ulp(a) - a$  $\Rightarrow$ $z = b_h + ulp(a)$
    \item $t = \circ (b-z)$ $\Rightarrow$ $t = b - (b_h+ ulp(a))$ we have that $b = b_h+b_{\ell}$ $\Rightarrow$ $t = b_{\ell} - ulp(a)$
\end{itemize}
So we have that $s+t = a+b_h+ulp(a)+b_{\ell}-ulp(a)$ $\Rightarrow$ $s+t$ is exactly equal to $a+b$.\\

We suppose $a > 0$ and $b > 0$ (respectively  $a < 0$ and $b < 0$).
We suppose that $\lvert b \rvert < ulp(a)$. Either we have $ 0 < \lvert b \rvert < 2^{-53}. \lvert a \rvert $ or  $ 2^{-53} \lvert a  \rvert \le \lvert b \rvert \le  2^{-52}. \lvert a \rvert$.\\
If $ 0 < \lvert b \rvert < 2^{-53}. \lvert a \rvert $ for the modes of Rounding $RNDZ$, $RNDN$ and $RNDD$ (respectively $RNDZ$, $RNDN$ and $RNDU$) then $b$ will be ignored
otherwise $RNDU$(respectively  $RNDD$) $b$ is not ignore, so  we have $\circ (b) =  ulp(a)$;\\
If $ 2^{-53} \lvert a  \rvert \le \lvert b \rvert \le  2^{-52}. \lvert a \rvert$ for the modes of Rounding $RNDZ$ and $RNDD$ (respectively $RNDZ$ and $RNDU$) 
then $b_{\ell}$ will be also ignored  for $RNDN$(closest round) and $RNDU$(respectively $RNDN$ and $RNDD$).\\

The first case when $b$ is ignored so we have :
\begin{itemize}
\item $s = \circ (a+b)$ $\Rightarrow$ $s = a$
\item $z = \circ (s-a)$ $\Rightarrow$ $z = a  - a$  $\Rightarrow$ $z = 0$
\item $t = \circ (b-z)$ $\Rightarrow$ $t = b - 0$  $\Rightarrow$ $t = b$
\end{itemize}
So we have that $s+t = a+b$ $\Rightarrow$ $s+t$ is exactly equal to $a+b$.\\

The second case when $b$ is not ignored so we have $\circ (b) = ulp(a)$.
\begin{itemize}
    \item $s = \circ (a+b)$ $\Rightarrow$ $s = a + ulp(a)$
    \item $z = \circ (s-a)$ $\Rightarrow$ $z = a  + ulp(a) - a$  $\Rightarrow$ $z = ulp(a)$
    \item $t = \circ (b-z)$ $\Rightarrow$ $t = b - ulp(a)$  
\end{itemize}
So we have that $s+t = a+ulp(a)+b-ulp(a)$ $\Rightarrow$ $s+t$ is exactly equal to $a+b$.\\

We suppose that $a >0$ and $b<0$ (respectively $a <0$ and $b>0$):\\
If $\lvert b \rvert \ge \lvert \frac{a}{2} \rvert$.\\
So we have:
\begin{itemize}
    \item $s = \circ (a+b)$, we have $\frac{-b}{2} \le a \le -2b$  after the \textbf{Sterbenz}'s lemma (lemma \ref{lem:ster}) , we have $s$ is exactly equal to $a+b$
    \item $z = \circ (s-a)$ $\Rightarrow$ $z =\circ ((a+b)-a)$ $\Rightarrow$ $z = \circ (b) = b$ $\Rightarrow$ $z=b$
    \item $t= \circ (b-z)$  $\Rightarrow$ $t = \circ (b-b)$ $\Rightarrow$ $t=0$
\end{itemize}
So we have, $s+t$ is exactly equal to $a+b$.\\

If $\lvert b \rvert < \lvert \frac{a}{2} \rvert$:
\begin{itemize}
    \item $s = \circ (a+b)$ $\Rightarrow$ $\frac{a}{2} < s \le \ a$ $\Rightarrow$ $\frac{a}{2} \le s \le \ 2a$
    \item $z = \circ (s-a)$ $\Rightarrow$ and  $\frac{a}{2} \le s \le \ 2a$ after the \textbf{Sterbenz}'s lemma (lemma \ref{lem:ster}) , we have $z$ is exactly equal to $s-a$.
    \item $t = \circ (b-z)$ $\Rightarrow$ $t = \circ (b - (s-a)$ $\Rightarrow$ $t = \circ(a+b -s)$  $\Rightarrow$ $t = \circ(a+(b-s))$ or $-\frac{a}{2} < b < 0$\\
    and $\frac{a}{2} \le s \le \ 2a$  $\Rightarrow$ $\frac{a}{2} -0 < s - b < 2a - (-\frac{a}{2})$ $\Rightarrow$ $\frac{a}{2}< s - b < \frac{3a}{2}$ $\Rightarrow$ $\frac{a}{2} \le s - b \le 2a$ after the \textbf{Sterbenz}'s lemma (lemma \ref{lem:ster}), we have $t$ is exactly equal to $a+b-s$.\\
\end{itemize}
We have $s+t = a+b-s+s$ $\Rightarrow$ $s+t = a+b$.\\
(By symetric, we have the same results for $a<0$ and $b>0$.)\\
The $4$ modes of rounding for $Add112$ have an exact equality between $s+t$ and $a+b$.\\

$(2)$:\\
According to Proof$(1)$ , We have $6$ possibilities for the result of $s$ and $t$:\\
\begin{itemize}

\item If $s = a+b$ and $t = 0$ so   $ \lvert t \rvert \le 2^{-53} \lvert s \rvert$ for  $RNDN$ and  $ \lvert t \rvert \le 2^{-52} \lvert s \rvert$ for the direct rounding modes.\\

\item If $s = a$ and $t =b$ ($a$ and $b$ have the same sign) with  $\lvert b \rvert < 2^{-53}.\lvert a \rvert$
$\Rightarrow$ $ \lvert t \rvert \le 2^{-53} \lvert s \rvert$ for $RNDN$ a fortiori  $ \lvert t \rvert \le 2^{-52} \lvert s \rvert$ for the direct rounding modes.\\

\item If $s=a+b_h$ and $t = b_l$ ($a$, $b_h$ and $b_{\ell}$ are the same sign). 
We suppose that $a >0$, $b_h > 0$ and $b_{\ell} > 0$.
We know that $0 \le  b_{\ell}   < 2^{-53}.  a $ 
$\Rightarrow$ $  a  >  2^{53}.  b_{\ell} $ and that $b_h$ is a multiple of 
$ulp(a)$ $\Rightarrow$ $  b_h  > 2.  b_{\ell} $. So we have $a + b_h > 2^{53}b_{\ell} + 2b_{\ell}$ $\Rightarrow$ $a + b_h > (2^{53} + 2).b_{\ell}$
or $2^{53} +2 > 2^{53}$ $\Rightarrow$ $a + b_h > 2^{53}b_{\ell}$ 
$\Rightarrow$ $s > 2^{53}.t $ $\Rightarrow$ $t \le 2^{-53}.s$ $\Rightarrow$ $ \lvert t \rvert \le 2^{-53} .\lvert s \rvert$ (and similarly if $a <0$, $b_h <0$ and $b_{\ell} < 0$).\\

\item If $s = a +ulp(a)$ and $t = b -ulp(a)$:\\
We know according to  proof($1$) that $\lvert b \rvert \le 2^{-53} \lvert a \rvert$.\\
We suppose that $a>0$ and $b>0$, so we have:\\
$t \le t+ulp(a)$ $\Rightarrow$ $t \le b - ulp(a) + ulp(a)$ $\Rightarrow$ $t \le b$ 
$\Rightarrow$ $\lvert t \rvert \le 2^{-53} \lvert a \rvert $\\
We search the upper bound of $\lvert a \rvert $ as a function of $\lvert s \rvert$:\\
$s = a+ ulp(a)$ $\Rightarrow$ $s \ge a$ $\Rightarrow$ $\lvert s \rvert \ge \lvert a \rvert $ so:\\
$\lvert t \rvert \le 2^{-53} \lvert s \rvert $ \\
(and similarly if $a<0$ and $b<0$).\\


\item If $s = a + b_h + ulp(a)$ and $t = b_{\ell} -ulp(a)$\\
We know according to  proof($1$) that $\lvert b_{\ell} \rvert \le 2^{-53} \lvert b_h \rvert$.\\
We suppose that $a>0$ and $b_h>0$, so we have:\\
$t \le t+ulp(a)$ $\Rightarrow$ $t \le b_{\ell} - ulp(a) + ulp(a)$ $\Rightarrow$ $t \le b_{\ell}$ 
$\Rightarrow$ $\lvert t \rvert \le 2^{-53} \lvert b_h\rvert $  $\Rightarrow$ $\lvert t \rvert \le 2^{-53} \lvert a +b_h\rvert $\\
We search the upper bound of $\lvert a + b_h \rvert $ as a function of $\lvert s \rvert$:\\
$s = a+ b_h +ulp(a)$ $\Rightarrow$ $s \ge a + b_h$ $\Rightarrow$ $\lvert s \rvert \ge \lvert a + b_h \rvert $ so:\\
$\lvert t \rvert \le 2^{-53} \lvert s \rvert $ \\
(and similarly if $a<0$ and $b<0$).\\

\item If $t = a+b -s$ , so  $\lvert t \rvert = \lvert a+b-s \rvert = \lvert s-(a+b) \rvert $  according to collary \ref{coroll:UX} $\Rightarrow$ :\\
$$\lvert s-(a+b) \rvert \le u.\lvert s \rvert$$ with $u$ is the \textbf{unit Roundoff}(see \ref{sub:U}).
\end{itemize}
For all cases,  we have $\lvert t \rvert \le u. \lvert s \rvert$ so for $RNDN$: $\lvert t \rvert \le 2^{-53}. \lvert s \rvert$ and for the others rounding modes: $\lvert t \rvert \le 2^{-52}. \lvert s \rvert$.
\end{proof}