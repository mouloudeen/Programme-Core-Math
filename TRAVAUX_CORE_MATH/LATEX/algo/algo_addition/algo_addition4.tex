\section{Addition Operators}
\subsection{Add133}
See algorithm \ref{algo:Add133}
\begin{lem}[Add133] Let $a$ a \textbf{double} number and $(b_h, b_m, b_{\ell})$ a   \textbf{triple-Double} number, $r_h$, $r_m$ and  $r_{\ell}$ result of $Add133(a,b_h,b_m,b_{\ell})$ for the $4$ modes of rounding, considering that there is no \textbf{overflow} so $ \lvert r_{\ell} \rvert \le u. \lvert r_m \rvert$, $\lvert r_m \rvert  \le u.\lvert  r_h \rvert$ and $\lvert r_{\ell} \rvert  \le u^2.\lvert  r_h \rvert$
\end{lem}

\begin{proof} \color{-yellow}
We suppose that $\lvert b_m \rvert \le u. \lvert b_h \rvert$, $\lvert b_{\ell} \rvert \le u. \lvert b_m \rvert$, according to  $Add133$,  so we have:
\begin{itemize}
\item $r_m+r_{\ell} = Add112(t_2, t_4)$ thanks to $Add112$ $\Rightarrow$ $r_m+r_{\ell} = t_2+ t_4$ and $ \lvert r_{\ell} \rvert \le u. \lvert r_m \rvert$. 

\item $t_4 = \circ(t_3 + b_{\ell})$ $\Rightarrow$ $t_4 = (t_3 + b_{\ell}).(1+\epsilon)$ with $\lvert \epsilon \rvert \le u$.

\item $t_2+ t_3 = Add112(t_1, b_m)$ thanks to $Add112$ $\Rightarrow$ $t_2+ t_3 = t_1+ b_m$ and $ \lvert t_3 \rvert \le u. \lvert t_2 \rvert$.

\item $r_h+t_1 = Add112(a,b_h)$ thanks to $Add112$ $\Rightarrow$ $r_h+t_1 = a+b_h$ and $ \lvert t_1 \rvert \le u. \lvert r_h \rvert$ $\Rightarrow$
\end{itemize}
$$r_m+r_{\ell} = t_2+ t_4$$
$$r_m+r_{\ell} = t_2 + (t_3 + b_{\ell}).(1+\epsilon)$$
$$r_m+r_{\ell} = t_2 + t_3.(1+\epsilon)+ b_{\ell}.(1+\epsilon)$$
$$r_m+r_{\ell} = t_2 + t_3+ t_3.\epsilon + b_{\ell}.(1+\epsilon)$$
As $t_2+t_3 = t_1+b_m$ $\Rightarrow$
$$r_m+r_{\ell} = t_1+b_m+ t_3.\epsilon + b_{\ell}.(1+\epsilon)$$
$$ \lvert r_m+r_{\ell} \rvert  = \lvert  t_1+b_m+ t_3.\epsilon + b_{\ell}.(1+\epsilon) \lvert$$
$$\lvert r_m+r_{\ell} \rvert  \le \lvert  t_1+b_m \rvert + \lvert t_3.\epsilon \rvert + \lvert b_{\ell}.(1+\epsilon) \lvert$$ 
We search the lower bound of $\lvert t_3 \rvert $ in function of $\lvert  t_1+b_m \rvert$:\\
$\lvert t_1 + b_m \rvert = \lvert t_2 + t_3 \rvert $ $\Rightarrow$ $\lvert t_1 + b_m \rvert \ge (1 + \frac{1}{u}).\lvert  t_3 \rvert $ $\Rightarrow$ $\lvert t_3 \rvert \le \frac{u}{u+1}.\lvert  t_1 + b_m \rvert $
$$\lvert r_m+r_{\ell} \rvert  \le \lvert  t_1+b_m \rvert +\frac{u}{u+1}. \lvert (t_1 + b_m) .\epsilon \rvert + \lvert b_{\ell}.(1+\epsilon) \lvert$$
$$\lvert r_m+r_{\ell} \rvert  \le \lvert  t_1+b_m \rvert +\frac{u^2}{u+1}. \lvert (t_1 + b_m) \rvert + \lvert b_{\ell}.(1+ u) \lvert$$
$$\lvert r_m+r_{\ell} \rvert  \le +\frac{u^2+u+1}{u+1}. \lvert (t_1 + b_m) \rvert + \lvert b_{\ell}.(1+ u) \lvert$$ 
According to the conditions of $Add112$, we have $\lvert t_1 \rvert \ge \lvert b_m \rvert $ and also that $\lvert b_{\ell} \rvert \le u. \lvert b_m \rvert $ $\Rightarrow$
$$\lvert r_m+r_{\ell} \rvert  \le \frac{2.(u^2+u+1)+u(u+1)^2}{u+1}.\lvert  t_1 \rvert$$
$$\lvert r_m+r_{\ell} \rvert  \le \frac{u^3+4.u^2+3.u+1}{u+1}.\lvert  t_1 \rvert$$
As $ \lvert t_1 \rvert \le u. \lvert r_h \rvert $ $\Rightarrow$
$$\lvert r_m+r_{\ell} \rvert  \le u.\frac{u^3+4.u^2+3.u+1}{u+1}.\lvert  r_h \rvert$$
 We have: $\frac{u^4+4.u^3+3.u^2+u}{u+1} \le u$ $\Rightarrow$
$$\lvert r_m+r_{\ell} \rvert  \le u.\lvert  r_h \rvert$$
We know that $ \lvert r_{\ell} \rvert \le u. \lvert r_m \rvert$ $\Rightarrow$ $(1 - u). \lvert r_m \rvert \le \lvert r_m + r_{\ell} \rvert $.\\
We have :
$$(1 - u).\lvert r_m \rvert  \le u.\lvert  r_h \rvert$$
$$\lvert r_m \rvert  \le \frac{u}{1-u}.\lvert  r_h \rvert$$
But $\frac{u}{1-u} \le u$ $\Rightarrow$
$$\lvert r_m \rvert  \le u.\lvert  r_h \rvert$$
and
$$\lvert r_{\ell} \rvert  \le u^2.\lvert  r_h \rvert$$
\end{proof}

\begin{theo}[Relative error algorithm $Add133$ without occuring of cancellation]

Let $a$ a \textbf{double} number and $(b_h,b_m,b_{\ell})$ a \textbf{triple-double} number are the arguments of the function $Add133$.\\
So:\\
$$r_h +r_m +r_{\ell} = (a + (b_h+ b_m +b_{\ell})).(1+\epsilon)$$ with $\lvert \epsilon \rvert \le 3.u^3$.
\label{the1:Add133}
\end{theo}

$3.u^3= 2^{-157.41}$ for $RNDN$ and $3.u^3 = 2^{-154.41}$ for the other rounding modes.
\begin{proof} \color{-yellow}
According to the calculation technique of the theorem $4.2$ proof's(\cite{lauter2005basic}).\\
We have $\lvert a \rvert \ge \lvert b_h \rvert$, either $a > 0$ and $b_h > 0$ or $a < 0$ and $b_h < 0$. As they're symmetric, we only use $a > 0$ and $b_h > 0$.\\
We have from Algorithme \ref{algo:Add133}:\\
$r_m + r_{\ell} = Add112(t_2,t_4)$ based on $Add112$  $\Rightarrow$ $r_m + r_{\ell} = t_2+t_4$\\
$t_4 = \circ(t_3+b_{\ell})$ $\Rightarrow$ $t_4 = (t_3 +b_{\ell})(1+\epsilon_1)$ with $\lvert \epsilon_1 \rvert \le u$\\
$t_2+ t_3 = Add112(t_1,b_m)$ based on $Add112$  $\Rightarrow$ $t_2+ t_3 = t_1+b_m$\\
$r_h + t_1 = Add112(a,b_h)$ based on $Add112$  $\Rightarrow$ $r_h + t_1 = a+b_h$\\
So we have:
$$r_m + r_{\ell} = t_2+(t_3 +b_{\ell})(1+\epsilon_1)$$
$$r_m + r_{\ell} = t_2+t_3 + t_3.\epsilon+ b_{\ell}.(1+\epsilon_1)$$
As $t_2 + t_3 = t_1 +b_m$ $\Rightarrow$
$$r_m + r_{\ell} = t_1 +b_m + t_3.\epsilon_1+ b_{\ell}.(1+\epsilon_1)$$
Then we have:
$$r_h + r_m + r_{\ell} = r_h +t_1 +b_m + t_3.\epsilon_1+ b_{\ell}.(1+\epsilon_1)$$
As $r_h +t_1 = a + b_h$
$$r_h + r_m + r_{\ell} = a + b_h +b_m + t_3.\epsilon_1+ b_{\ell}.(1+\epsilon_1)$$
$$r_h + r_m + r_{\ell} = a + (b_h +b_m + b_{\ell}+ t_3.\epsilon_1+ b_{\ell}.\epsilon_1$$
$$r_h + r_m + r_{\ell} = a + (b_h +b_m + b_{\ell})+ \delta$$
with $\delta = t_3.\epsilon_1+ b_{\ell}.\epsilon_1 $\\
So:
$$ \lvert \delta \rvert \le \lvert t_3.\epsilon_1 \rvert +\lvert b_{\ell}.\epsilon_1 \rvert $$
We know that $t_1 + b_m = t_2 + t_3$ and that $ \lvert t_3 \rvert \le u.\lvert t_2 \rvert $ $\Rightarrow$ 
$$ \lvert t_1 + b_m \rvert  \ge (\frac{1}{u}+1). \lvert t_3 \rvert $$
$$ \lvert t_3 \rvert  \le \frac{u}{u+1}. \lvert t_1 + b_m  \rvert $$
$$ \lvert t_3 \rvert  \le \frac{u}{u+1}. (\lvert t_1 \rvert + \lvert b_m  \rvert) $$
As $a + b_h = r_h + t_1$ and that $ \lvert t_1 \rvert \le u.\lvert r_h \rvert $ $\Rightarrow$ 
$$ \lvert a + b_h  \rvert  \ge (\frac{1}{u}+1). \lvert t_1 \rvert $$
$$ \lvert t_1 \rvert  \le \frac{u}{u+1}. \lvert a + b_h  \rvert $$
$\Rightarrow$
$$ \lvert \delta \rvert \le \frac{u}{u+1}.(\frac{u}{u+1}. \lvert a + b_h  \rvert +  \lvert b_m \rvert).\epsilon_1 \rvert +\lvert b_{\ell}.\epsilon_1 \rvert $$
As $\lvert \epsilon_1 \rvert \le u$, $ \lvert b_m \rvert \le u.\lvert b_h \rvert \le u.\lvert a + b_h \rvert $ and $\lvert b_{\ell} \rvert \le u.\lvert b_m \rvert$ $\Rightarrow$
$$ \lvert \delta \rvert \le \frac{u}{u+1}.(\frac{u}{u+1}. \lvert a + b_h  \rvert + u .\lvert  a+b_h \rvert).u  +u.u^2 .\lvert  a+b_h \rvert $$

After calculating, we find:\\
$$ \lvert \delta \rvert \le \frac{u^5+3.u^4+3.u^3}{(u+1)^2}.\lvert a + b_h\rvert $$
We search the upper bound of $\lvert a +b_h \rvert $ in function of $\lvert a+b_h +b_m + b_{\ell} \rvert $:\\
first, we calculate the upper bound of $\lvert b_m + b_{\ell} \rvert$ in function of $\lvert a +b_h \rvert$:
$$\lvert b_m + b_{\ell} \rvert \le  \lvert b_m \rvert + \lvert b_{\ell} \rvert  $$
$$\lvert b_m + b_{\ell} \rvert \le  u.\lvert a +b_h \rvert + u^2.\lvert a +b_h\rvert  $$

We deduce:
$$\lvert a+b_h +b_m + b_{\ell} \rvert \ge (1-( u +u^2)).\lvert a + b_h \rvert $$
$$\lvert a+b_h +b_m + b_{\ell} \rvert \ge \lvert (1-u -u^2) \rvert .\lvert a +b_h \rvert $$

$$ \frac{1}{(1- u -u^2)}.\lvert a+b_h +b_m + b_{\ell} \rvert \ge  \lvert a \rvert $$
$\Rightarrow$
$$ \lvert \delta \rvert \le \frac{u^5+3.u^4+3.u^3}{(u+1)^2.(1-u-u^2)}.\lvert a+b_h +b_m + b_{\ell} \rvert $$
$$ \lvert \delta \rvert \le \lvert \epsilon \rvert.\lvert a+b_h +b_m + b_{\ell} \rvert $$
We have $\frac{u^5+3.u^4+3.u^3}{(u+1)^2.(1-u-u^2)} \le 3.u^3$ $\Rightarrow$ $\lvert \epsilon \rvert \le  3.u^3 $
\end{proof}

\begin{theo}[ Relative error algorithm $Add133$ with a bounded cancellation]
Let $a$ a \textbf{double} number and ($b_h$, $b_m$, $b_{\ell}$) a \textbf{triple-double} number . We have for the algorithm $Add133$ with
$a$ and ($b_h$, $b_m$, $b_{\ell}$) it's arguments.\\
If $a$ and $b_h$ are different sign and we suppose $\lvert b_h \rvert \le 2^{-\mu} \lvert a \rvert $ with $\mu \ge 1$.\\
So :
$$r_h +r_m +r_{\ell} = (a+ (b_h+ b_m +b_{\ell})).(1+\epsilon)$$
with $$\lvert \epsilon \rvert \le  \frac{u^5+3.u^4+3.u^3}{(u+1)^2} .\frac{1 - 2^{-\mu}}{1 -(1+u+u^2).2^{-\mu}} \le 6.u^3$$
\end{theo}

\begin{proof} \color{-yellow}
According to the calculation technique of the theorem $4.3$ proof's(\cite{lauter2005basic}).\\
We suppose $\lvert b_h \rvert \le 2^{-\mu} \lvert a \rvert$ with $\mu \ge 1$.
First we look for the upper bound, using the results of the proof of Theorem \ref{the1:Add133}. We have:\\
$$\lvert b_{\ell} \rvert \le u.\lvert b_m \rvert  \le u^2.\lvert b_h \rvert\le u^2.2^{- \mu}\lvert a \rvert $$\\
then we search the lower bound of $\lvert a+b_h \rvert $ in function of $\lvert a \rvert$:\\
$$\lvert a + b_h \rvert \ge (1 -2^{- \mu}). \lvert a\rvert$$
Now, we search the upper bound of $\lvert b_m \rvert$ and $\lvert b_h \rvert$ in function of $\lvert a+b_h \rvert $.\\
$$\lvert b_m  \rvert \le \frac{u.2^{- \mu}}{1- 2^{-\mu}}\lvert a + b_h\rvert$$
and 
$$\lvert b_{\ell}  \rvert \le \frac{u^2.2^{- \mu}}{1- 2^{-\mu}}\lvert a + b_h\rvert$$

As :
$$ \lvert \delta \rvert \le \frac{u^5+3.u^4+3.u^3}{(u+1)^2}.\lvert a + b_h\rvert $$
Now, we search the lower bound for $\lvert a + b_h + b_m + b_{\ell} \rvert$ depending on $\lvert a + b_h \rvert$.\\
We have:
$$\lvert b_m + b_{\ell} \rvert \le \lvert b_m \rvert + \lvert b_{\ell} \rvert$$
$$\lvert b_m + b_{\ell} \rvert \le (u+u^2).2^{-\mu}.\lvert a  \rvert$$
$$\lvert b_m  + b_{\ell} \rvert \le \frac{(u+u^2).2^{-\mu}}{1 - 2^{-\mu}}\lvert a + b_h \rvert$$
$$\lvert a + b_h + b_m + b_{\ell} \ge  \lvert a_h + b_h \rvert \frac{1 - 2^{-\mu} - (u+u^2).2^{-\mu}}{1 - 2^{-\mu}}$$
So we have for $\lvert \delta \rvert $:
$$ \lvert \delta \rvert \le \lvert a + b_h + b_m + b_{\ell} \rvert. \frac{u^5+3.u^4+3.u^3}{(u+1)^2} .\frac{1 - 2^{-\mu}}{1 -(1+u+u^2).2^{-\mu}} $$

So:
$$\lvert \epsilon \rvert \le  \frac{u^5+3.u^4+3.u^3}{(u+1)^2} .\frac{1 - 2^{-\mu}}{1 -(1+u+u^2).2^{-\mu}}$$
As $\mu \ge 1$, we want the upper bound of $\frac{1 - 2^{-\mu}}{1 -(1+u+u^2).2^{-\mu}}$ $\Rightarrow$
$$\frac{1}{1/2} \le 2$$
So :
$$\lvert \epsilon \rvert \le 6.u^3$$
\end{proof}
