\section{Multiplication Operators}

\subsection{Mul112}
See algorithm \ref{algo:Mul112}
\begin{lem}[Mul112]
Let $a$ and $b$  floating point numbers,  $s$ and $t$ result of $Mul112(a,b)$ for the $4$ modes of rounding, considering that there is no \textbf{overflow} so :
\begin{itemize}
 \item  $(1)$ $r_1+r_2$ is exactly equal to $a.b$.              
\item  $(2)$ $ \lvert r_2 \rvert \le u.\lvert r_1 \rvert $
\end{itemize} 
\end{lem}

\begin{proof} \color{-yellow}
$(1)$
According to \cite{muller2010handbook},  $FMA(a,b,-c)$ is exacty equal to $a+b-c$ for the $4$ modes of rounding .\\
Based on Algorithm $Mul112$, we have :\\
$r_2 = FMA(a,b,-r1)$ that supposed $\Rightarrow$ $r_2 = a*b -r_1$\\
So 
$$r_1 + r_2 = a*b-r_1 +r_1$$
$$r_1 + r_2 = a*b$$

$(2)$ 
$r1 = \circ (a*b)$ according to the collary \ref{coroll:UX}
$\Rightarrow$
$r1 =  a*b +\epsilon$ with $\lvert \epsilon \rvert \le u.\lvert r_1 \rvert$
We know that $r2$ is exactly equal to $a*b -r_1$.\\
$\Rightarrow$ $r_2 = a*b -r1$ $\Rightarrow$
$$r_2 = a*b -(a*b+\epsilon)$$
$$r_2 = -\epsilon$$
That $\lvert \epsilon \rvert \le u.\lvert r_1 \rvert$ $\Rightarrow$ $\lvert r_2 \rvert \le u.\lvert r_1 \rvert $.\\
\end{proof}