\chapter*{Introduction}
\addcontentsline{toc}{chapter}{Introduction}.
This thesis aims to explain in a mathematical way the steps for the implementation of the logarithm with correct rounding on floating numbers.
This function is one of the functions of the COREMATH\footnote{\url{https://core-math.gitlabpages.inria.fr}} project. \\
This project implements mathematical functions with correct rounding to be able to integrate into mathematical libraries for the new revision of the
standard $IEEE 754$.\\
\ \\
The definition of correct rounding given by the $IEEE 754$ standard is as follows \enquote{Given a mathematical function $f$ and a floating point number $x$,
the correct rounding of $f(x)$ is the floating number $y$ closest to $f(x)$ according to the given rounding mode (nearest, towards zero, towards $-\infty$ or towards
$+\infty$)}. This standard imposes the correct rounding for the four elementary arithmetic operations which are addition, subtraction,
multiplication and division. But it does not impose for mathematical functions. For now, there is no mathematical library that gives us
exactly the correct rounding.\\
\ \\
This project already has the implementation of this logarithm for single precision ($binary32$ format of $ IEEE 754$). The calculation steps of our logarithm will be in
function of double precision ($binary64$ format of $IEEE 754$). We will use basic floating point algorithms which are \textbf{FastSum} and
\textbf{DEKKER-PRODUCT}. \\
\ \\
My research paper will be devited into five chapters. \\
In the first chapter, we explain floating point and the $IEEE754$ standard with some definitions. Then we integrate some arithmetic tools that will be used for calculations in chapters $4$, and $5$. Then we give some notation rules.\\
\ \\
In the second chapter, we detail the steps of the $cr\log$. first, we talk about the special cases. Then, we explain the argument reduction with the
algorithms of \textbf{Tang} and \textbf{Gal}. After, we calculate and evaluate the approximation polynomial thanks to the formula of \textbf{Taylor}.\\
\ \\
For the third chapter, we define the addition and multiplication algorithms which will give us results in \textbf{Double-Double} type and
calculate their relative errors for each of these functions.\\
\ \\
In the fourth chapter, we define other addition and multiplication algorithms which give us results in \textbf{Triple-Double} type and
we calculate their relative errors.\\
\ \\
The last chapter explain how the logarithm will unfold with its three $cr\log$.